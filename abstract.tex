\documentclass[10pt,twocolumn,twoside]{IEEEtran}

\usepackage{amsmath}
\usepackage{bm}
\usepackage{nicefrac}
\usepackage{booktabs}
\usepackage{array}
\usepackage{multirow}
\usepackage{threeparttable}
\usepackage{makecell}
\usepackage[procnumbered,ruled,vlined,linesnumbered]{algorithm2e}
\usepackage{siunitx}
\usepackage{stfloats}
\usepackage{graphicx}
\usepackage{subfigure}
\usepackage{hyperref}
\usepackage{array}
\usepackage{epstopdf}
\usepackage{balance}
\usepackage{tabularx}
\usepackage{stfloats}
\usepackage{lipsum}

\newtheorem{problem}{Problem}
\newtheorem{theorem}{Theorem}[section]
\newtheorem{corollary}[theorem]{Corollary}
\newtheorem{lemma}[theorem]{Lemma}
\newtheorem{definition}[theorem]{Definition}
\newtheorem{fact}[theorem]{Fact}

% \newcommand{\proof}{{\bf Proof.}\hskip 0.3truecm}
% \newcommand{\endproof}{\quad \(\Box\)}

\newcommand{\defeq}{\stackrel{\mathrm{def}}{=}}
\newcommand{\setof}[1]{\left\{#1 \right\}}
% \newcommand{\sizeof}[1]{\left|#1 \right|}
\newcommand{\abs}[1]{\left|#1 \right|}
\newcommand{\mean}[1]{\mathbb{E}[#1]}
\newcommand{\ceil}[1]{\left\lceil#1\right\rceil}
\newcommand{\Abs}[1]{\left\Vert#1\right\Vert}
\newcommand{\trace}[1]{\mathrm{Tr}\left(#1\right)}

\newcommand{\bsym}[1]{\boldsymbol{#1}}
\newcommand{\myord}[1]{{#1}^{\rm{th}}}
\newcommand{\rea}{\mathbb{R}}
\newcommand{\gr}{\mathcal{G}}
\newcommand{\vecd}{\bsym{d}}
\newcommand{\veca}{\bsym{a}}
\newcommand{\vecb}{\bsym{b}}
\newcommand{\vece}{\bsym{e}}
\newcommand{\allone}{\bsym{1}}
\newcommand{\vecl}{\bsym{l}}
\newcommand{\vecv}{\bsym{v}}
\newcommand{\vecpi}{\bsym{\pi}}
\newcommand{\vecx}{\bsym{x}}

\newcommand{\pscr}{\mathscr{P}}

\newcommand{\lap}{\bsym{L}}
\newcommand{\matd}{\bsym{D}}
\newcommand{\mata}{\bsym{A}}
\newcommand{\matb}{\bsym{B}}
\newcommand{\matw}{\bsym{W}}
\newcommand{\matp}{\bsym{P}}
\newcommand{\matpcal}{\bsym{\mathcal{P}}}
\newcommand{\matf}{\bsym{F}}
\newcommand{\matfstar}{\bsym{F}^*}
\newcommand{\mati}{\bsym{I}}
\newcommand{\matq}{\bsym{Q}}
\newcommand{\matr}{\bsym{R}}
\newcommand{\mats}{\bsym{S}}
\newcommand{\matx}{\bsym{X}}
\newcommand{\maty}{\bsym{Y}}
\newcommand{\matz}{\bsym{Z}}
\newcommand{\matpi}{\bsym{\Pi}}

\newcommand{\lemref}[1]{Lemma~\ref{#1}}
\newcommand{\thmref}[1]{Theorem~\ref{#1}}
\newcommand{\probref}[1]{Problem~\ref{#1}}
\newcommand{\algoref}[1]{Algorithm~\ref{#1}}
\newcommand{\defref}[1]{Definition~\ref{#1}}
\newcommand{\secref}[1]{Section~\ref{#1}}
\newcommand{\tabref}[1]{Table~\ref{#1}}
\newcommand{\figref}[1]{Figure~\ref{#1}}

\newcommand{\edge}[2]{\langle #1, #2 \rangle}


\DeclareMathOperator*{\argmin}{arg\,min}
\DeclareMathOperator*{\argmax}{arg\,max}

\DontPrintSemicolon
\SetKw{KwAnd}{and}
\SetFuncSty{textsc}
\SetKwInOut{Input}{Input\ \ \ \ }

\SetKwInOut{Output}{Output}
\newcommand{\biophoto}[1]{\includegraphics[width=1in,height=1.25in,clip,keepaspectratio]{#1}}
\newcommand{\todo}[1]{{\bf \color{red} TODO: #1}}

\begin{document}

\title{Means of Hitting Times for Random Walks on Graphs: Connections, Computation, and Optimization}
\author{Haisong~Xia,
    %22210240101@m.fudan.edu.cn    
    Wanyue~Xu~\IEEEmembership{Student Member,~IEEE},
    Zuobai~Zhang,
    %Zhuoqing~Song,
    Zhongzhi~Zhang~\IEEEmembership{Member,~IEEE}
    \thanks{This work was  supported  by the National Natural Science Foundation of China (No. U20B2051) and the Shanghai Municipal Science and Technology
        Major Project  (No. 2021SHZDZX03). Any correspondence should be addressed to Zhongzhi Zhang.
    }


    \thanks{
        Haisong Xia, Wanyue Xu,  Zuobai~Zhang, and Zhongzhi Zhang are with the Shanghai Key Laboratory of Intelligent Information Processing, School of Computer Science, Fudan University, Shanghai 200433, China; Zhongzhi Zhang is also with the Shanghai Engineering Research Institute of Blockchains, Fudan University, Shanghai 200433, China; and Research
        Institute of Intelligent Complex Systems, Fudan University, Shanghai 200433, China. {\tt\small zhangzz@fudan.edu.cn}
    }
    \thanks{
        An earlier version of this paper was presented in part at the Thirteenth ACM International Conference on Web Search and Data Mining (WSDM)~\cite{ZhXuZh20} [DOI: 10.1145/3336191.3371777].
    }

}

\markboth{IEEE Transactions on Information Theory}
% {Xia \MakeLowercase{\textit{et al.}}: Absorbing Time of Random Walks as a Node Group Centrality}
{Xia \MakeLowercase{\textit{et al.}}: Means of Hitting Times for Random Walks on Graphs: Connections, Computation, and Optimization}

\IEEEtitleabstractindextext{
    \begin{abstract}
        For random walks on graph $\gr$ with $n$ vertices and $m$ edges, the mean hitting time $H_j$ from a vertex chosen from the stationary distribution to vertex $j$ measures the importance for $j$, while the Kemeny constant $K$ is the mean hitting time from one vertex to another selected randomly according to the stationary distribution. In this paper, we first establish a connection between the two quantities, representing $K$ in terms of $H_j$ for all vertices. We then develop an efficient algorithm estimating $H_j$ for all vertices and $K$ in nearly linear time of $m$. Moreover, we extend the centrality $H_j$ of a single vertex to $H(S)$ of a vertex set $S$, and establish a link between $H(S)$ and some other quantities. We further study the NP-hard problem of selecting a group $S$ of $k\ll n$ vertices with minimum $H(S)$, whose objective function is monotonic and supermodular. We finally propose two greedy algorithms approximately solving the problem. The former has an approximation factor $(1-\frac{k}{k-1}\frac{1}{e})$ and $O(kn^3)$ running time, while the latter returns a $(1-\frac{k}{k-1}\frac{1}{e}-\epsilon)$-approximation solution in nearly-linear time of $m$, for any parameter $0<\epsilon <1$. Extensive experiment results validate the performance of our algorithms.
    \end{abstract}

    \begin{IEEEkeywords}
        Random walk, hitting time, Kemeny constant, spectral algorithm, complex network, optimization, vertex centrality.
    \end{IEEEkeywords}
}

\maketitle

\IEEEdisplaynontitleabstractindextext

\IEEEpeerreviewmaketitle

\bibliographystyle{IEEEtran}
\balance
\bibliography{refs}

\end{document}